\chapter{Аналитический раздел}
\label{cha:analysis}
В данном разделе будет рассмотрено описание алгоритмов поиска расстояния Левенштейна и Дамерау-Левенштейна\cite{Lowenstein}.
\par Допустимы следующие редакторские операции:
\begin{itemize}
	\item M - совпадение, штраф -- 0;
	\item I - вставка, штраф -- 1;
	\item R - замена, штраф -- 1;
	\item D - удаление, штраф -- 1;
\end{itemize}
\par Пусть $ S_{1} $  и $ S_{2} $ -- строки длиной M и N соответственно над некоторым алфавитом, тогда расстояние Левенштейна можно подсчитать по рекуррентной формуле \cite{habr}:
\begin{equation}
D(i, j) = \begin{cases}
j, \quad i = 0\\
i, \quad j = 0, i > 0\\
min = (\\
D(i,j-1) + 1,\\
D(i-1, j) + 1,\\
D(i-1, j-1)+m(S_{1}[i],S_{2}[j])\\
)
\end{cases}
\end{equation}

\begin{flushleft}
	где \(m(S_{1}[i],S_{2}[j])\) равно нулю, если \(S_{1}[i]=S_{2}[j]\) и единице в противном случае.
\end{flushleft}
\par При поиске расстояние Дамерау-Левенштейна добавлена операция транспозиции, штраф которой равен 1, в связи с чем оно может быть вычислено по формуле:
\[ D(i, j) =  \left\{
\begin{aligned}
	&j, && i = 0\\
	&i, && j = 0, i > 0\\		    	
	&min \left\{
	\begin{aligned}
	&D(i, j - 1) + 1,\\
	&D(i - 1, j) + 1,\\
	&D(i - 1, j - 1) + m(S_{1}[i], S_{2}[i]), \\
	&D(i - 2, j - 2) + m(S_{1}[i], S_{2}[i]),\\
	\end{aligned} \right.
	&& 
	\begin{aligned}
	&, \text{ если } i, j > 1 \\
	& \text{ и } S_{1}[i] = S_{2}[j - 1] \\
	& \text{ и } S_{1}[i - 1] =  S_{2}[j] \\
	\end{aligned} \\ 
	&min \left\{
	\begin{aligned}
	&D(i, j - 1) + 1,\\
	&D(i - 1, j) + 1, \\
	&D(i - 1, j - 1) + m(S_{1}[i], S_{2}[i]),\\
	\end{aligned} \right.  &&, \text{иначе}
\end{aligned} \right.
\]
\paragraph{Вывод:} были рассмотрены алгоритмы поиска расстояний Левенштейна и Дамерау-Левенштейна.
