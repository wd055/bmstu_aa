\chapter{Технологический раздел}
\label{cha:impl}
В данном разделе будут рассмотрены требования к программному обеспечению, средства реализации, представлен листинг кода.
\section{Средства реализации}
В данной работе используется язык программирования Python, в связи с тем, что имею большой опыт работы с ним. Среда разработки Visual Studio Code.\\
Для замера процессорного времени используется функция procees\_time() из библиотеки time.

\section{Листинг кода}
В листингах 3.1-3.4 приведены алгоритмы поиска расстояния Левенштейна и Дамерау-Левенштейна.
\begin{lstlisting}[caption= Матричный алгоритм поиска расстояния Левенштейна]
def levenstein_m(origin, target):
	l1 = len(origin) + 1
	l2 = len(target) + 1
	matr = [[0 for i in range(l2)] for i in range(l1)]
	
	for i in range(1, l1):
		matr[i][0] = i
	for i in range(1, l2):
		matr[0][i] = i
	
	for i in range(1, l1):
		for j in range(1, l2):
			matr[i][j] = min(matr[i - 1][j] + 1, 
							matr[i][j - 1] + 1,
							matr[i - 1][j - 1] + (origin[i - 1] != target[j - 1]))
	
	return matr[l1 - 1][l2 - 1]
\end{lstlisting}

\begin{lstlisting}[caption= Рекурсивный алгоритм поиска расстояния Левенштейна]
def _rec(origin, l1, target, l2):
	if not l1:
		return l2
	elif not l2:
		return l1
	
	a1 = _rec(origin, l1-1, target, l2) + 1
	a2 = _rec(origin, l1, target, l2-1) + 1
	a3 = _rec(origin, l1-1, target, l2-1) + \
			(origin[l1 - 1] != target[l2 - 1])
	
	return min(a1, a2, a3)

def levenstein_r(origin, target):
	l1 = len(origin)
	l2 = len(target)
	
	return _rec(origin, l1, target, l2)

\end{lstlisting}

\begin{lstlisting}[caption= Матрично-рекурсивный алгоритм поиска расстояния Левенштейна]
def _rec(s1, s2, matr, i, j):
	if not i:
		matr[i][j] = j
	elif not j:
		matr[i][j] = i
	
	elif matr[i][j] == -1:
		matr[i][j] = min(_rec(s1, s2, matr, i - 1, j) + 1, 
						_rec(s1, s2, matr, i, j - 1) + 1,
						_rec(s1, s2, matr, i - 1, j - 1) + int(s1[i - 1] != s2[j - 1]))
	
	return matr[i][j]


def levenstein_rm(origin, target):
	l1 = len(origin) + 1
	l2 = len(target) + 1
	matr = [[-1 for i in range(l2)] for i in range(l1)]
	
	_rec(origin, target, matr, l1 - 1, l2 - 1)
	return matr[l1 - 1][l2 - 1]
\end{lstlisting}

\begin{lstlisting}[caption= Матричный алгоритм поиска расстояния Дамерау-Левенштейна]
def dl_matrix(origin, target):
	l1 = len(origin) + 1
	l2 = len(target) + 1
	matrix = [[0 for i in range(l2)] for i in range(l1)]
	
	for i in range(1, l1):
		matrix[i][0] = i
	for i in range(1, l2):
		matrix[0][i] = i
	
	for i in range(1, l1):
		for j in range(1, l2):
			if (i > 1 and j > 1) and (origin[i - 2] == target[j - 1] and origin[i - 1] == target[j - 2]):
				switch = matrix[i - 2][j - 2] + 1
				matrix[i][j] = min(matrix[i - 1][j] + 1,
								matrix[i][j - 1] + 1,
								matrix[i - 1][j - 1] + (origin[i - 1] != target[j - 1]),
								switch)
			else:                    
				matrix[i][j] = min(matrix[i - 1][j] + 1,
								matrix[i][j - 1] + 1,
								matrix[i - 1][j - 1] + (origin[i - 1] != target[j - 1]))
	
	return matrix[l1 - 1][l2 - 1]
\end{lstlisting}

\section{Оценка затрат памяти}
В таблице 3.1 приведены объёмы памяти, затрачиваемые различными типами данных в языке Python.
\begin{table}[H]
	\caption{Память, потребляемая разными типами данных в Python}
	\begin{tabular}{|c|c|}
		\hline
		\textbf{Структура данных} & \textbf{Занимаемая память(байт)} \\ \hline
		Целое число               & 14                               \\ \hline
		Пустой список             & 36                               \\ \hline
		Список с 1 элементом      & 40                               \\ \hline
		Пустая строка             & 25                               \\ \hline
		Строка длиной 4           & 29                               \\ \hline
	\end{tabular}
\end{table}
В таблицах 3.2-3.4 приведены оценки памяти, затрачиваемой на работу алгоритмов поиска расстояния Левенштейна.
\begin{table}[H]
	\caption{Память, потребляемая в матричном алгоритме поиска расстояния Левенштейна}
	\begin{tabular}{|c|c|}
		\hline
		\textbf{Структура данных}                                                    & \textbf{Занимаемая память(байт)}                                                                        \\ \hline
		Матрица                                                                      & \begin{tabular}[c]{@{}c@{}}36+(len(s1)+1)*(36+4*(len(s2)+1))+\\ (len(s1)+1)*(len(s2)+1)*14\end{tabular} \\ \hline
		\begin{tabular}[c]{@{}c@{}}2 вспомогательные \\ переменные(int)\end{tabular} & 28                                                                                                      \\ \hline
		2 счётчика(int)                                                              & 28                                                                                                      \\ \hline
		передача параметров                                                          & 2*(25+len(s))                                                                                            \\ \hline
	\end{tabular}
\end{table}
В матричном алгоритме Дамерау-Левенштейна используется аналогичное количество памяти, однако на 1 вспомогательную переменную больше.
\par Для рекурсивного и матрично-рекурсивного алгоритмов поиска расстояния Левенштейна также будет оцениваться память при максимальной глубине рекурсивного вызова $n$ равной длине большего слова.
\begin{table}[H]
	\caption{Память, потребляемая в рекурсивном алгоритме поиска расстояния Левенштейна}
	\begin{tabular}{|c|c|}
		\hline
		\textbf{Структура данных} & \textbf{Занимаемая память(байт)} \\ \hline
		3 переменные(int)         & 52                               \\ \hline
		передача параметров       & 2*(25+len(s))+2*14               \\ \hline
		максимальная              & n*(52+2*(25+len(s))+2*14)        \\ \hline
	\end{tabular}
\end{table}
\begin{table}[H]
	\caption{Память, потребляемая в матрично-рекурсивном алгоритме поиска расстояния Левенштейна}
	\begin{tabular}{|c|c|}
		\hline
		\textbf{Структура данных} & \textbf{Занимаемая память(байт)}                                                                        \\ \hline
		матрица                   & \begin{tabular}[c]{@{}c@{}}36+(len(s1)+1)*(36+4*(len(s2)+1))+\\ (len(s1)+1)*(len(s2)+1)*14\end{tabular} \\ \hline
		передача параметров       & \begin{tabular}[c]{@{}c@{}}2*(25+len(s))+2*14+\\ 36+(len(s1)+1)*4\end{tabular}                          \\ \hline
		максимальная              &\begin{tabular}[c]{@{}c@{}}n*(28+28+36+(len(s1)+1)*(36+4*(len(s2)+1))+\\ (len(s1)+1)*(len(s2)+1)*14)\end{tabular}         \\ \hline
	\end{tabular}
\end{table}
Используя таблицы 3.2-3.4 можно оценить память, затрачиваемую на вычисление расстояния между двумя словами, длиной 10 символов.
\begin{table}[H]
	\caption{Память, потребляемая алгоритмами вычисления редакторского расстояния для двух строк длиной 10 символов}
	\begin{tabular}{|c|c|}
		\hline
		\textbf{Алгоритм}                                                          & \textbf{\begin{tabular}[c]{@{}c@{}}Затрачиваемая память\\ (байт)\end{tabular}} \\ \hline
		Матричный Левенштейна                                                      & 2736                                                                           \\ \hline
		Рекурсивный Левенштейна                                                    & 1500                                                                        \\ \hline
		\begin{tabular}[c]{@{}c@{}}Матрично-рекурсивный\\ Левенштейна\end{tabular} &      26660                                                                    \\ \hline
		\begin{tabular}[c]{@{}c@{}}Матричный\\ Дамерау-Левенштейна\end{tabular}    & 2750                                                                           \\ \hline
	\end{tabular}
\end{table}
Из таблицы 3.5 видно, что рекурсивный алгоритм потребляет наименьшее количество памяти для поиска расстояния между словами длиной 10 символов
\section{Проведение тестирования:}
Проведём тестирование программы по методу чёрного ящика. В столбцах "Ожидаемый результат" и "Полученный результат" находится 4 числа, соответствующие матричному, рекурсивному, матрично-рекурсивному алгоритмам поиска расстояния Левенштейна и матричному алгоритму поиска расстояния Дамерау-Левенштейна.
\begin{table}[H]
	\caption{Тестирование программы}
	\begin{tabular}{|c|c|c|}
		\hline
		\textbf{Входные данные} & \textbf{Ожидаемый результат} & \textbf{Полученный результат} \\ \hline
		\text{,}                  & 0 0 0 0                      & 0 0 0 0                       \\ \hline
		, abcd               & 4 4 4 4                      & 4 4 4 4                       \\ \hline
		abcd,               & 4 4 4 4                      & 4 4 4 4                       \\ \hline
		telo, stolb          & 3 3 3 3                      & 3 3 3 3                       \\ \hline
		abcd, cbef           & 3 3 3 3                      & 3 3 3 3                       \\ \hline
		abcd, bacd          & 2 2 2 1                      & 2 2 2 1                       \\ \hline
		1234, 5678           & 4 4 4 4                      & 4 4 4 4                       \\ \hline
	\end{tabular}
\end{table}
Все тесты пройдены успешно.
