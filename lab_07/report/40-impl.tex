\chapter{Технологический раздел}
\label{cha:impl}
В данном разделе будут представлены листинги кода реализованных алгоритмов и проведено тестирование.

\section{Средства реализации}
В данной работе используется язык программирования Python. Среда разработки Visual Studio Code. Для замера процессорного времени используется функция perf\_counter() из библиотеки time.
\par Замеры времени были произведены на: Intel(R) Core(TM) i5-8250U CPU @1.60GHz 1.80 Ghz, 4 ядра, 8 логических процессоров.
\par Для формирования словаря была использована библиотека pymorphy2 \cite{pymorphy}.

\section{Листинг кода}
В листинге \ref{code:fullsearch} приведен код алгоритма полного перебора.
\begin{lstlisting}[caption= Алгоритм полного перебора, label=code:fullsearch]
def fullsearch(key, dictionary):
	for record in dictionary:
		if record[0] == key:
			return record
	return None
\end{lstlisting}

\par В листинге \ref{code:binsearch} представлен код двоичного поиска.
\begin{lstlisting}[caption= Муравьиный алгоритм, label=code:binsearch]
def binsearch(key, dictionary):
	lb = 0
	rb = len(dictionary) - 1
	while lb <= rb :
		mid = (lb + rb) // 2
		if dictionary[mid][0] == key:
			return dictionary[mid]
		elif dictionary[mid][0] < key:
			lb = mid + 1
		else:
			rb = mid - 1
	return None
\end{lstlisting}

\par В листинге \ref{code:segmentation} представлен код сегментации и поиска в сегментах.
\begin{lstlisting}[caption= Класс муравья, label=code:segmentation]
def statseg(dictionary):
	segments = []
	for record in dictionary:
		added = False
		for i in range(len(segments)):
			if record[0][0] == segments[i][0][0]:
			segments[i][1].append(record)
			added = True
		if not added:
			segments.append([[record[0][0]], [record]])
	segments.sort(key=lambda record: len(record[1]), reverse=True)
	
	i = len(segments) - 1
	while i > 0:
		if len(segments[i][1]) < 100:
			for letter in segments[i][0]:
				segments[i - 1][0].append(letter)
			for word in segments[i][1]:
				segments[i - 1][1].append(word)
			segments.pop(i)
		i -= 1
	segments.sort(key=lambda record: len(record[1]), reverse=True)
	for segment in segments:
		segment[1].sort(key=lambda record: record[0])
	return segments


def segmentation(key, segments):
	for segment in segments:
		if key[0] in segment[0]:
			return binsearch(key, segment[2])
\end{lstlisting}

\section{Тестирование}
Были проведены тесты с первым, последним, средним словом и словом, отсутствующим в словаре.
\begin{table}[H]
	\caption{Результаты тестирования}
	\label{fig:restable}
\begin{tabular}{|c|c|c|c|l|}
	\hline
	\textbf{}                                                               & \textbf{ёкнуть}      & \textbf{абажур}   & \textbf{пролетать}      & ---  \\ \hline
	\textbf{\begin{tabular}[c]{@{}c@{}}Полный \\ перебор\end{tabular}}      & {[}'ёкнуть', 6978{]} & {[}'абажур', 0{]} & {[}'пролетать', 4768{]} & None \\ \hline
	\textbf{\begin{tabular}[c]{@{}c@{}}Двоичный \\ поиск\end{tabular}}      & {[}'ёкнуть', 6978{]} & {[}'абажур', 0{]} & {[}'пролетать', 4768{]} & None \\ \hline
	\textbf{\begin{tabular}[c]{@{}c@{}}С сегменти-\\ рованием\end{tabular}} & {[}'ёкнуть', 6978{]} & {[}'абажур', 0{]} & {[}'пролетать', 4768{]} & None \\ \hline
\end{tabular}
\end{table}
Все тесты пройдены успешно.

\section{Вывод}
В данном разделе были рассмотрены листинги кода реализованных алгоритмов и проведено тестирование.