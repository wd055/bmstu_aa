\chapter{Аналитический раздел}
\label{cha:analysis}
В данном разделе будет представлено понятие умножения матрицы, рассмотрен алгоритм Винограда и способы его параллельной реализации.

\par Матрицей $A$ размера $m\times n$ называется прямоугольная таблица чисел, функций или алгебраических выражений, содержащая $m$ строк и $n$ столбцов \cite{Belousov}. Умножение матриц возможно, только когда число столбцов первой матрицы равно числу строк второй.

\section{Алгоритм Винограда.}
При рассмотрении результата умножения матриц видно, что каждый элемент -- скалярное произведение векторов, представляющих строки и столбцы матриц \cite{Algolib}.
\par Рассмотрим 2 вектора $V = (v1, v2, v3, v4)$ и $W =(w1,w2,w3,w4)$. Их скалярное произведение равно:
\begin{equation}
	V\bullet W=v1w1+w2w2+v3w3+v4w4
\end{equation}
Или:
\begin{equation}
	V\bullet W=(v1+w2)(v2+w1)+(v3+w4)(v4+w3)-v1v2-v3v4-w1w2-w3w4
\end{equation}
\par Можно заметить, что выражение в правой части последнего равенства допускает предварительную обработку: его части можно вычислить заранее и запомнить для каждой строки первой матрицы и для каждого столбца второй. На практике это означает, что над предварительно обработанными элементами нам придется выполнять лишь первые два умножения и последующие пять сложений, а также дополнительно два сложения.

\section{Параллельный алгоритм Винограда.}
При использовании алгоритма Винограда можно заметить 2 основные части, которые могут подойти для распараллеливания, это предварительный подсчёт двух массивов, которые используется в дальнейшем, при этом они сами не обращаются к собственным элементам, которые можно подсчитывать параллельно. Значит, каждый поток может вычислять некоторую часть данных массивов
\par Другой частью, пригодной для параллельных вычислений является главный цикл, в котором происходит построчное заполнение результирующей матрицы, следует заметить, что значение каждой строки не зависит от остальных, таким образом каждый поток может выполнять вычисление отдельной строки.

\section{Параллельное программирование.}
Параллельные вычисления - способ организации компьютерных вычислений, при котором программы разрабатываются, как набор взаимодействующих вычислительных процессов, работающих асинхронно и при этом одновременно \cite{Parallel}.

Параллельное программирование - это техника программирования, которая использует преимущества многоядерных или многопроцессорных компьютеров и является подмножеством более широкого понятия многопоточности (multithreading).

Использование параллельного программирования становится наиболее необходимым, поскольку позволяет максимально эффективно использовать возможности многоядерных процессоров и многопроцессорных систем. По ряду причин, включая повышение потребления энергии и ограничения пропускной способности памяти, увеличивать тактовую частоту современных процессоров стало невозможно. Вместо этого производители процессоров стали увеличивать их производительность за счёт размещения в одном чипе нескольких вычислительных ядер, не меняя или даже снижая тактовую частоту. Поэтому для увеличения скорости работы приложений теперь следует по-новому подходить к организации кода, а именно - оптимизировать программы под многоядерные системы.

\section{Вывод}
В данном разделе был рассмотрен алгоритм Винограда, способы его распараллеливания и понятие параллельного программирования.