\chapter{Исследовательский раздел}
\label{cha:research}
В данном разделе будет измерено время работы алгоритмов и сделаны выводы на основе полученных данных.
\section{Сравнительный анализ на основе замеров времени работы алгоритмов}
Для каждого из алгоритмов был проведён замер времени на квадратных матрицах размерами от 100 до 1000, с шагом 100, матрицы заполнены случайными элементами. Для каждого теста было проведено 5 замеров, на графиках представлен усреднённый результат. На рисунках \ref{par1}--\ref{com} время, обозначаемое t представлено в миллисекундах.
\par На рисунке \ref{par1} представлено сравнение времени работы реализации алгоритма умножения матриц Винограда с параллельным главным циклом. Числами обозначено количество выделенных потоков.

\begin{figure}[H]
	\centering
	\begin{tikzpicture}[scale=1.5]
		\begin{axis}[
			axis lines = left,
			xlabel = {Длина массива},
			ylabel = {$t$},
			legend pos=north west,
			ymajorgrids=true,
			clip=false
			]
			\addplot[color=red, mark=square] table[x index=0, y index= 1] {src/tempRes/parallel1_1}; 
			\addplot[color=green, mark=square] table[x index=0, y index= 1] {src/tempRes/parallel1_2}; 
			\addplot[color=yellow, mark=square] table[x index=0, y index= 1] {src/tempRes/parallel1_4}; 
			\addplot[color=pink, mark=square] table[x index=0, y index= 1] {src/tempRes/parallel1_8}; 
			\addplot[color=black, mark=square] table[x index=0, y index= 1] {src/tempRes/parallel1_16}; 
			\addplot[color=blue, mark=round] table[x index=0, y index= 1] {src/tempRes/parallel1_32}; 
			
			\addlegendentry{1}
			\addlegendentry{2}
			\addlegendentry{4}
			\addlegendentry{8}
			\addlegendentry{16}
			\addlegendentry{32}
		\end{axis}
	\end{tikzpicture}
	\caption{Сравнение времени работы алгоритма Винограда с параллельным главным циклом}
	\label{par1}
\end{figure}
\par Из графика \ref{par1} становится ясно, что прирост производительности имеет место только при увеличении числа потоков до 4, после чего она меняется незначительно.

\par На рисунке \ref{par2} представлено сравнение времени работы реализации алгоритма умножения матриц Винограда с параллельным вычислением массивов mulH и mulV. Числами обозначено количество выделенных потоков.
\begin{figure}[H]
	\centering
	\begin{tikzpicture}[scale=1.5]
		\begin{axis}[
			axis lines = left,
			xlabel = {Длина массива},
			ylabel = {$t$},
			legend pos=north west,
			ymajorgrids=true,
			clip=false
			]
			\addplot[color=red, mark=square] table[x index=0, y index= 1] {src/tempRes/parallel2_1}; 
			\addplot[color=green, mark=square] table[x index=0, y index= 1] {src/tempRes/parallel2_2}; 
			\addplot[color=yellow, mark=square] table[x index=0, y index= 1] {src/tempRes/parallel2_4}; 
			\addplot[color=pink, mark=square] table[x index=0, y index= 1] {src/tempRes/parallel2_8}; 
			\addplot[color=black, mark=square] table[x index=0, y index= 1] {src/tempRes/parallel2_16}; 
			\addplot[color=blue, mark=round] table[x index=0, y index= 1] {src/tempRes/parallel2_32}; 
			
			\addlegendentry{1}
			\addlegendentry{2}
			\addlegendentry{4}
			\addlegendentry{8}
			\addlegendentry{16}
			\addlegendentry{32}
		\end{axis}
	\end{tikzpicture}
	\caption{Сравнение времени работы алгоритма Винограда с параллельным вычислением mulV и mulH}
	\label{par2}
\end{figure}
\par Из графика \ref{par2} видно, что распараллелиливание вычисления массивов mulV и mulH не имеет смысла, так как практически уменьшает затраты по времени.

\par На рисунке \ref{com} представлено сравнение времени работы последовательной реализации алгоритма Винограда и его параллельных реализаций. На графике введены обозначения: <<последовательный>> -- последовательный алгоритм Винограда, <<Main>> -- параллельно выполняется главный цикл, <<Mul>> -- параллельно считаются массивы mulV и mulH, цифры рядом - количество выделенных потоков.
\begin{figure}[H]
	\centering
	\begin{tikzpicture}[scale=1.5]
		\begin{axis}[
			axis lines = left,
			xlabel = {Длина массива},
			ylabel = {$t$},
			legend pos=north west,
			ymajorgrids=true,
			clip=false
			]
			\addplot[color=red, mark=square] table[x index=0, y index= 1] {src/tempRes/linear}; 
			\addplot[color=green, mark=square] table[x index=0, y index= 1] {src/tempRes/parallel1_1}; 
			\addplot[color=yellow, mark=square] table[x index=0, y index= 1] {src/tempRes/parallel1_2}; 
			\addplot[color=pink, mark=square] table[x index=0, y index= 1] {src/tempRes/parallel1_4}; 
			\addplot[color=black, mark=square] table[x index=0, y index= 1] {src/tempRes/parallel2_1}; 
			\addplot[color=blue, mark=square] table[x index=0, y index= 1] {src/tempRes/parallel2_2}; 
 
			\addlegendentry{Последовательный}
			\addlegendentry{Main 1}
			\addlegendentry{Main 2}
			\addlegendentry{Main 4}
			\addlegendentry{Mul 1}
			\addlegendentry{Mul 2}
		\end{axis}
	\end{tikzpicture}
	\caption{Сравнение времени работы алгоритма Винограда и его параллельных реализаций}
	\label{com}
\end{figure}
\par Из рисунка \ref{com} становится ясно, что наиболее эффективным является умножение матриц алгоритмом Винограда с параллельным главным циклом, в то время как использование параллельных вычислений при подсчёте массивов mulV и mulH не даёт результатов по сравнению с последовательной реализацией.

\section{Вывод} 
В итоге, можно сказать о том, что параллельное вычисление главного цикла в алгоритме Винограда является самым эффективным способом уменьшения затрачиваемого времени из рассмотренных.
