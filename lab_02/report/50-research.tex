\chapter{Исследовательский раздел}
\label{cha:research}
В данном разделе будет измерено время работы алгоритмов и сделаны выводы на основе полученных данных.
\section{Сравнительный анализ на основе замеров времени работы алгоритмов}
Для каждого из алгоритмов был проведён замер времени на матрицах размером от 100 до 1000 с шагом 100. Для каждого размера было проведено 10 повторений.
\par На рисунке \ref{fig:matrixEven} представлено сравнение времени для алгоритмов при чётных размерах матриц.
\begin{figure}
	\centering
	\begin{tikzpicture}[scale=1.5]
	\begin{axis}[
	axis lines = left,
	xlabel = {Размер матрицы},
	ylabel = {Время (секунды)},
	legend pos=north west,
	ymajorgrids=true
	]
	\addplot[color=red, mark=square] table[x index=0, y index= 1] {src/standardEven.txt}; 
	\addplot[color=green, mark=square] table[x index=0, y index= 1] {src/winogradEven.txt}; 
	\addplot[color=blue, mark=square] table[x index=0, y index= 1] {src/winoptimEven.txt}; 

	\addlegendentry{Стандартный}
	\addlegendentry{Алг. Винограда}
	\addlegendentry{Оптимизированный алг.}

	\end{axis}
	\end{tikzpicture}
	\caption{Сравнение времени работы алгоритмов на матрицах чётного размера}
	\label{fig:matrixEven}
\end{figure}

\par На рисунке \ref{fig:matrixOdd} представлено сравнение времени работы для алгоритмов при нечётных размерах матриц.
\begin{figure}[H]
	\centering
	\includegraphics[width=0.7\linewidth]{src/odd}
	\caption{Сравнение времени работы алгоритмов на матрицах нечётного размера}
	\label{fig:matrixOdd}
\end{figure}

\par Из рисунков \ref{fig:matrixOdd} и \ref{fig:matrixEven} видно, что для всех трёх алгоритмов асимптотика роста общая, однако как для чётных, так и для нечётных матриц оптимизированный алгоритм Винограда является самым быстрым, следующий по скорости алгоритм Винограда и самый медленный -- стандартный алгоритм умножения матриц.
\par Таким образом, данные, полученные в результате проведённых экспериментов подтверждают корректность рассчитанных ранее трудоёмкостей алгоритмов.

\section{Вывод}
В итоге, можно сказать о том, что оптимизированный алгоритм Винограда является самым эффективным из рассмотренных.
