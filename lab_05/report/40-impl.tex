\chapter{Технологический раздел}
\label{cha:impl}
В данном разделе будут представлены листинги кода реализованных алгоритмов и дано описание получаемых результатов работы программы.

\section{Средства реализации}
В данной работе используется язык программирования С++. Среда разработки Visual Studio Code. Для замера процессорного времени используется функция QueryPerformanceCounter из библиотеки windows.h. Параллельные вычисления выполняются с использование библиотеки <thread> \cite{threads}, а также применяются mutex из библиотеки <mutex>.
\par Замеры времени были произведены на: Intel(R) Core(TM) i5-8250U CPU @1.60GHz 1.80 Ghz, 4 ядра, 8 логических процессоров.

\section{Листинг кода}
В листинге \ref{code:xor} приведен код алгоритма XOR-шифрования.
\begin{lstlisting}[caption= Алгоритм XOR-шифра, label=code:xor]
int xorCrpyt(char *string, int l, bitset<8> keyToEncrypt)
{
	for (int i = 0; i < l; i++)
	string[i] ^= keyToEncrypt.to_ulong();
	return 0;
}
\end{lstlisting}

\par В листинге \ref{code:caesar} представлен код алгоритма шифра Цезаря.
\begin{lstlisting}[caption= Алгоритм шифра Цезаря, label=code:caesar]
int caesarCrypt(char *string, int l, int shift)
{
	for (int i = 0; i < l; i++)
	string[i] = (string[i] - 97 + shift) % 26 + 97;
	return 0;
}
\end{lstlisting}

\par В листингах \ref{code:part1}-\ref{code:part3} представлен код 3 этапов выполнения конвейера.
\begin{lstlisting}[caption= Первый этап выполнения конвейера, label=code:part1]
void Conveyor::part1()
{
	while (this->ft1 < this->taskNumber)
	{
		Request *req;
		if (this->startingQ.size())
		{
			req = this->startingQ.front();
			this->startingQ.pop();
		}
		else
		continue;
		
		req->s1 = GetTimestamp();
		
		caesarCrypt(req->string, req->l, 12);
		
		req->e1 = GetTimestamp();
		
		this->ft1++;
		
		this->m1.lock();
		this->q2.push(req);
		this->m1.unlock();
	}
}
\end{lstlisting}
\begin{lstlisting}[caption= Второй этап выполнения конвейера, label=code:part2]
void Conveyor::part2()
{
	while (this->q2.size() == 0)
	continue;
	
	while (this->ft2 < this->taskNumber)
	{
		while (this->q2.size() == 0)
		continue;
		
		Request *req;
		
		this->m1.lock();
		req = this->q2.front();
		this->q2.pop();
		this->m1.unlock();
		
		req->s2 = GetTimestamp();
		bitset<8> key('e');
		
		xorCrpyt(req->string, req->l, key);
		req->e2 = GetTimestamp();
		
		this->ft2++;
		
		this->m2.lock();
		this->q3.push(req);
		this->m2.unlock();
	}
}
\end{lstlisting}
\begin{lstlisting}[caption= Третий этап выполнения конвейера, label=code:part3]
void Conveyor::part3()
{
	while (this->q3.size() == 0)
	continue;
	
	while (this->ft3 < this->taskNumber)
	{
		while (this->q3.size() == 0)
		continue;
		
		Request *req;
		this->m2.lock();
		req = this->q3.front();
		this->q3.pop();
		this->m2.unlock();
		
		req->s3 = GetTimestamp();
		
		bitset<8> key('k');
		xorCrpyt(req->string, req->l, key);
		req->e3 = GetTimestamp();
		
		this->ft3++;
		this->result.push_back(req);
	}
}	
\end{lstlisting}
\par В листинге \ref{code:run} представлен код основного потока и генератора заявок.
\begin{lstlisting}[caption= Третий этап выполнения конвейера, label=code:run]
void Conveyor::run()
{
	generateRequests();
	
	thread t1 = thread(&Conveyor::part1, this);
	thread t2 = thread(&Conveyor::part2, this);
	thread t3 = thread(&Conveyor::part3, this);
	
	t1.join();
	t2.join();
	t3.join();
}

void Conveyor::generateRequests()
{
	for (int i = 0; i < taskNumber; i++)
	{
		Request *req = new Request(taskLength);
		req->number = i;
		stringGenerator(req->string, req->l);
		startingQ.push(req);
	}
	cout << startingQ.size();
}
\end{lstlisting}

\section{Собираемые данные}
Результатом работы программы является массив зашифрованных строк, создаваемых случайно, все строки равной длины и данные о времени начала и конца обработки каждой из заявок в 1, 2 и 3 этапах. Зная эту информацию, мы можем получить ифнормацию о максимальном, минимальном, среднем времени во 2 и 3 очередях и в системе вообще.

\section{Вывод}
В данном разделе были рассмотрены листинги кода реализованных алгоритмов и конвейера, описаны результаты работы программы.