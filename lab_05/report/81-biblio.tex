\begin{thebibliography}{9}
    \addcontentsline{toc}{chapter}{Литература}
    \label{cha:biblio}
    
    \bibitem{conveyorbelt} Конвейер (значения)[Электронный ресурс]. —2010. —Режим досту­па: https://dic.academic.ru/dic.nsf/ruwiki/141054 (дата обращения:25.11.2021).
    
    \bibitem{conveyor} Конвейеризация как средство повышения производительно­сти ЭВМ[Электронный ресурс]. — 2021. — Режим доступа: https://spravochnick.ru/informatika/konveyerizaciya_kak_sredstvo_povysheniya_proizvoditelnosti_evm/ (дата обращения: 25.11.2021).
    
    \bibitem{crypting} Malenkovich Serge. Зачем нужно шифрование?[Электронныйресурс]. — 2013. — Режим доступа: https://www.kaspersky.ru/blog/encryption-reasons/879/ (дата обращения: 25.11.2021).
    
    \bibitem{cr_algs} malkoran. Элементарные шифры на понятном языке[Электронный ре­сурс]. —2019. —Режим доступа: https://habr.com/ru/post/444176/ (датаобращения: 25.11.2021).
    
    \bibitem{threads} Varun. C++11 Multithreading [Электронный ре­сурс]. — 2015. — Режим доступа: https://thispointer.com//c-11-multithreading-part-1-three-different-ways-to-create-threads/(дата обращения: 25.11.2021).16

\end{thebibliography}
