\chapter{Аналитический раздел}
\label{cha:analysis}
В данном разделе будет представлено понятие конвейерных вычислений, рассмотрены способы шифрования строк.

\section{Конвейер и конвейерная обработка}
Конвейер — машина непрерывного транспорта, предназначенная для перемещения сыпучих, кусковых или штучных грузов \cite{conveyorbelt}.
\par Конвейеризация в обобщённом смысле базируется на разделении выполняемой операции на более мелкие составляющие, которые называются подфункциями, и предоставлении для выполнения каждой подфункции своего аппаратного блока \cite{conveyor}.

\par Вычислительный конвейер предполагает перемещение команд или данных по этапам цифрового вычислительного конвейера со скоростью, не зависящей от протяжённости конвейера (количества этапов), а зависит только от скорости подачи информации на конвейерные этапы. Скорость задаётся временем, в течение которого один компонент вычислительной операции способен пройти каждый этап, то есть самой большой задержкой на этапе, который выполняет отдельный участок функции. Это также значит, что скорость вычислений задаётся и скоростью поступления информации на вход конвейера. 
\par В случае, когда какая-либо функция при её обычном выполнении реализуется за временной интервал T, но имеется возможность её деления на поочерёдное исполнение N подфункций, то в идеальном конвейере, если вычисление этой функции повторяется многократно, возможно её исполнение за временной период Т/N, то есть в N раз увеличить производительность. Различие реального и идеального конвейера заключается в наличии в реальной вычислительной системе различных помех. Общий смысл помехи заключается в присутствии фактора, который связан с самой функцией, конструктивными особенностями конвейера или его применения, препятствующих постоянному приходу новой информации на конвейерные этапы с самой большой скоростью.

\section{Шифрование строк}
Шифрование –- это преобразование информации, делающее ее нечитаемой для посторонних. При этом  доверенные лица могут провести дешифрование и прочитать исходную информацию. Существует множество способов шифрования/дешифрования, но секретность данных основана не на тайном алгоритме, а на том, что ключ шифрования (пароль) известен только доверенным лицам \cite{crypting}.
\par Существуют разные алгоритмы шифрования \cite{cr_algs}.
\begin{itemize}
	\item Шифр Атбаша -- алфавит зеркально отражается, то есть "A" становится "Z" и так далее.
	\item Шифр Цезаря -- имеется ключ в виде числа от 1 до 25(для латиницы) и каждая буква алфавита смещается вправо или влево на ключевое число значений.
	\item Шифр Вернама(XOR-шифр) -- Сообщение разбиваем на отдельные символы и каждый символ представляем в бинарном виде. После чего посимвольно применим операцию XOR с ключом, в результате чего получается зашифрованное сообщение.
	\item Шифр кодового слова -- выбирается ключевое слово, все неповторяющиеся буквы из него выписываются в начало алфавита, остальной алфавит сдвигается.
\end{itemize}

\section{Вывод}
В данном разделе было представлено понятие конвейерных вычислений, рассмотрены способы шифрования строк.