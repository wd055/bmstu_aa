\chapter{Исследовательский раздел}
\label{cha:research}
В данном разделе будут приведены результаты работы программы и сделаны выводы.

\section{Анализ результатов}
Замеры времени проводились на конвейере, обрабатывающем 200 заявок, каждая из которых содержит строку длиной 1000000 символов. Было замерено время 20 конвейеров, приведены усреднённые результаты. Время t приводится в миллисекундах.

\par На графиках \ref{tin2q} и \ref{tin3q} представлено время проведённое каждой заявкой во 2-ой и 3-ей очередях. 
\begin{figure}[H]
	\centering
	\begin{tikzpicture}[scale=1.5]
		\begin{axis}[
			axis lines = left,
			xlabel = {Номер заявки},
			ylabel = {$t$},
			legend pos=north east,
			ymajorgrids=true,
			clip=false
			]
			\addplot[color=black] table[x index=0, y index= 1] {src/timein2Q}; 
			
			\addlegendentry{Во второй очереди}
		\end{axis}
	\end{tikzpicture}
	\caption{Время проведённое заявкой во 2-ой и 3-ей очередях}
	\label{tin2q}
\end{figure}

\begin{figure}[H]
	\centering
	\begin{tikzpicture}[scale=1.5]
		\begin{axis}[
			axis lines = left,
			xlabel = {Номер заявки},
			ylabel = {$t$},
			legend pos=north east,
			ymajorgrids=true,
			clip=false
			]
			\addplot[color=black] table[x index=0, y index= 1] {src/timein3Q}; 
			
			\addlegendentry{В третьей очереди}
		\end{axis}
	\end{tikzpicture}
	\caption{Время проведённое заявкой в 3-ей очереди}
	\label{tin3q}
\end{figure}
\par На графике \ref{tinc} представлено время, проведённое каждой заявкой во всей системе.
\begin{figure}[H]
	\centering
	\begin{tikzpicture}[scale=1.2]
		\begin{axis}[
			axis lines = left,
			xlabel = {Номер заявки},
			ylabel = {$t$},
			legend pos=north east,
			ymajorgrids=true,
			clip=false
			]
			\addplot[color=black] table[x index=0, y index= 1] {src/wholetime}; 
			
			\addlegendentry{Суммарное время}
		\end{axis}
	\end{tikzpicture}
	\caption{Время проведённое в конвейере}
	\label{tinc}
\end{figure}
Из графиков видно, что время нахождения во 2-ой очереди резко растёт только в начале обработки, затем спадает и держится на низком уровне. В третьей очереди и всей системе в общем ситуация схожа, однако пики выше.
В таблице \ref{t:1} представлен минимальные, максимальные и средние времени, проведенного заявками в очередях и системе вообще.
\begin{table}[]
	\begin{tabular}{|l|l|l|l|}
		\hline
		& min      & max     & avg    \\ \hline
		2-ая оч. & 0.000295 & 0.73859 & 0.14   \\ \hline
		3-я оч.  & 0.00034  & 0.01242 & 0.0021 \\ \hline
		Система  & 7.7533   & 8.73173 & 7.89   \\ \hline
	\end{tabular}
\caption{Анализ данных}
\label{t:1}
\end{table}
\par В результате, можно сказать, что наибольшее время потрачено в ожидании поступления на конвейер, а значит первый этап является наиболее затратным по времени, что замедляет всю работу.

\section{Вывод} 
В данном разделе были рассмотрены результаты работы программы, стало ясно, что первый этап - шифрование шифром Цезаря замедляет работу всей системы, а также, что разница во времени работы 2 и 3 этапов крайне низка, что следует из низкого времени, проведенного в 3-ей очереди.
