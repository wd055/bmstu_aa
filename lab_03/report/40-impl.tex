\chapter{Технологический раздел}
\label{cha:impl}
В данном разделе будут представлены листинги кода реализованных алгоритмов.
\section{Средства реализации}
В данной работе используется язык программирования С++. Среда разработки Visual Studio Code. Для замера процессорного времени используется функция QueryPerformanceCounter из библиотеки windows.h.
\section{Листинг кода}
В листингах 3.1--3.3 приведены коды алгоритмов сортировки пузырьком с флагом, сортировки выбором и пирамидальной.
\begin{lstlisting}[caption= Сортировка пузырьком с флагом]
void bubbleSort(array_t array)
{
	bool changed = false;
	for (int i = 1; i < array.length; i++)
	{
		for (int j = 0; j < array.length - i; j++)
			if (array.array[j + 1] < array.array[j])
			{
				const int temp = array.array[j];
				array.array[j] = array.array[j + 1];
				array.array[j + 1] = temp;
				changed = true;
			}
		if (!changed)
			break;
	}
}
\end{lstlisting}

\begin{lstlisting}[caption= Сортировка выбором]
void selectionSort(array_t array)
{
	for(int i=0; i<array.length-1; i++)
	{
		int min_i = i;
		for(int j=i+1;j<array.length;j++)
		{
			if (array.array[j] < array.array[min_i])
			min_i = j;
		}
		if (min_i != i)
		{
			int tmp = array.array[i];
			array.array[i] = array.array[min_i];
			array.array[min_i] = tmp;
		}        
	}
}
\end{lstlisting}

\begin{lstlisting}[caption= Пирамидальная сортировка и необходимая для её работы функция siftDown]
void siftDown(array_t array, int i)
{
	int nMax = i;
	int value = array.array[i];
	while (true)
	{
		i = nMax;
		int childN = i*2 + 1;
		if ((childN < array.length) && (array.array[childN] > value))
			nMax = childN;
		childN++;
		if ((childN < array.length) && (array.array[childN] > array.array[nMax]))
			nMax = childN;
		if (nMax == i)
			break;
		array.array[i] = array.array[nMax];
		array.array[nMax] = value;
	}
}

void heapSort(array_t array)
{
	for (int i = array.length / 2 -1; i >= 0; --i)
		siftDown(array, i);
	while (array.length > 1)
	{
		array.length--;
		
		int firstElement = array.array[0];
		array.array[0] = array.array[array.length];
		array.array[array.length] = firstElement;
		
		siftDown(array, 0);
	}
}
\end{lstlisting}

\section{Проведение тестирования}
\label{sec:tests}
Изначально было проведено тестирование алгоритма сортировки пузырьком на массивах: [1,2,1,1,1], [5,4,3,2,1], [1,2,3,4,5], [5,1,2,4,3], все тесты были пройдены успешно.
\par После чего была проведена серия тестов на массивах длиной от 1 до 10. Сначала выполнялась сортировка пузырьковым алгоритмом, результат которого принимался за контрольное значение, после чего результаты сортировки алгоритмом выбора и пирамидальной сортировки сравнивались с контрольным значением и при совпадении результат считался корректным. Массивы заполнялись случайным образом и в возрастающем порядке. Все тесты были пройдены успешно.
%%%% mode: latex
%%%% TeX-master: "rpz"
%%%% End:
