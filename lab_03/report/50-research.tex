\chapter{Исследовательский раздел}
\label{cha:research}
В данном разделе будет измерено время работы алгоритмов и сделаны выводы на основе полученных данных.
\section{Сравнительный анализ на основе замеров времени работы алгоритмов}
Для каждого из алгоритмов был проведён замер времени на массивах длиной от 5 до 50000 тысяч, с шагом 5000, для каждой длины были проведены измерения на массивах, отсортированных в убывающем, возрастающем порядках и заполненных случайными элементами. Для каждого теста было проведено 10 замеров, на графиках представлен усреднённый результат.
\par На рисунке \ref{rls:randomTime} представлено сравнение времени работы алгоритмов при случайно заполненных массивах. На рисунках \ref{rls:randomTime}--\ref{rls:downTime} время, обозначаемое t представлено в миллисекундах.
\begin{figure}[H]
    \centering
    \begin{tikzpicture}[scale=1.5]
    \begin{axis}[
    axis lines = left,
    xlabel = {Длина массива},
    ylabel = {$t$},
    legend pos=north west,
    ymajorgrids=true,
    clip=false
    ]
    \addplot[color=red, mark=square] table[x index=0, y index= 1] {src/tempRes/boubbleSortRandom.txt}; 
    \addplot[color=green, mark=square] table[x index=0, y index= 1] {src/tempRes/selectionSortRandom.txt}; 
    \addplot[color=blue, mark=square] table[x index=0, y index= 1] {src/tempRes/heapSortRandom.txt}; 
    
    \addlegendentry{Пузырьком с флагом}
    \addlegendentry{Выбором}
    \addlegendentry{Пирамидальная}
    \end{axis}
    \end{tikzpicture}
    \caption{Сравнение времени работы алгоритмов при случайно заполненных массивах}
    \label{rls:randomTime}
\end{figure}

\par На рисунке \ref{rls:upTime} представлено сравнение времени работы алгоритмов при массивах, заполненных в возрастающем порядке.
\begin{figure}[H]
	\centering
	\begin{tikzpicture}[scale=1.5]
	\begin{axis}[
	axis lines = left,
	xlabel = {Длина массива},
	ylabel = {$t$},
	legend pos=north west,
	ymajorgrids=true
	]
	\addplot[color=red, mark=square] table[x index=0, y index= 1] {src/tempRes/boubbleSortSortedA.txt}; 
	\addplot[color=green, mark=square] table[x index=0, y index= 1] {src/tempRes/selectionSortSortedA.txt}; 
	\addplot[color=blue, mark=square] table[x index=0, y index= 1] {src/tempRes/heapSortSortedA.txt}; 
	
	\addlegendentry{Пузырьком с флагом}
	\addlegendentry{Выбором}
	\addlegendentry{Пирамидальная}
	\end{axis}
	\end{tikzpicture}
	\caption{Сравнение времени работы алгоритмов при массивах отсортированных в возрастающем порядке}
    \label{rls:upTime}
\end{figure}

\par На рисунке \ref{rls:downTime} представлено сравнение времени работы алгоритмов при массивах, заполненных в убывающем порядке.
\begin{figure}[H]
	\centering
	\begin{tikzpicture}[scale=1.5]
	\begin{axis}[
	axis lines = left,
	xlabel = {Длина массива},
	ylabel = {$t$},
	legend pos=north west,
	ymajorgrids=true
	]
	\addplot[color=red, mark=square] table[x index=0, y index= 1] {src/tempRes/boubbleSortSortedD.txt}; 
	\addplot[color=green, mark=square] table[x index=0, y index= 1] {src/tempRes/selectionSortSortedD.txt}; 
	\addplot[color=blue, mark=square] table[x index=0, y index= 1] {src/tempRes/heapSortSortedD.txt}; 
	
	\addlegendentry{Пузырьком с флагом}
	\addlegendentry{Выбором}
	\addlegendentry{Пирамидальная}
	\end{axis}
	\end{tikzpicture}
	\caption{Сравнение времени работы алгоритмов при массивах отсортированных в убывающем порядке}
    \label{rls:downTime}
\end{figure}

\par Из рисунков \ref{rls:randomTime}--\ref{rls:downTime} видно, что наиболее эффективным по времени в случаях с случайно заполненными массивами и массивами, заполненными в убывающем порядке оказался алгоритм пирамидальной сортировки, в то время как сортировка пузырьком с флагом самая быстрая при массивах, отсортированных в возрастающем порядке.
\par Таким образом, данные, полученные в результате проведённых экспериментов подтверждают корректность рассчитанных ранее трудоёмкостей алгоритмов.
\par\textbf{Вывод} 
\par В итоге, можно сказать о том, что пирамидальная сортировка является наиболее эффективной по времени среди рассмотренных алгоритмов.
