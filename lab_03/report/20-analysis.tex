\chapter{Аналитический раздел}
\label{cha:analysis}
В данном разделе будет представлено понятие сортировки, рассмотрены некоторые алгоритмы.

Алгоритм сортировки - алгоритм упорядочивания элементов в списке. Исходными данными для задачи сортировки является последовательность чисел $\left\langle a_1,a_2,\dots,a_n \right\rangle$. Результатом должна стать последовательность $\left\langle a_1',a_2',\dots,a_n' \right\rangle$, состоящая из тех же чисел, идущих в неубывающем порядке \cite{Cormen}. 

\par\textbf{Сортировка пузырьком с флагом}
\par Принцип действий заключается в обходе массив от начала до конца, попутно меняя местами неотсортированные соседние элементы. В результате первого прохода на последнее место будет перемещён максимальный элемент. После каждой итерации длина неотсортированной части массива уменьшается на 1. Алгоритм переноса максимального элемента в конец продолжается до тех пор, пока длина неотсортированной части не станет равной 1, или пока на очередной итерации не произойдёт ни одного обмена, за контроль чего и отвечает флаг.
\par\textbf{Сортировка выбором}
\par На очередной итерации будем находить минимум в массиве после текущего элемента и менять его с ним, если надо. Таким образом, после i-ой итерации первые i элементов будут стоять на своих местах. Нужно отметить, что эту сортировку можно реализовать двумя способами – сохраняя минимум и его индекс или просто переставляя текущий элемент с рассматриваемым, если они стоят в неправильном порядке. Первый способ оказался немного быстрее, поэтому он и реализован.
\par\textbf{Пирамидальная сортировка}
\par Идея данного алгоритма сортировки заключается в построении возрастающей пирамиды -- почти заполненного двоичного дерева, в котором значение каждого элемента больше или равно значения всех его потомков \cite{archive}. Для сортировки применяется алгоритм просеивания вниз, которая перемещает выбранный элемент на такую позицию, в которой он не нарушает свойство возрастающей пирамиды. Просеивание вниз:
\begin{enumerate}[1)]
	\item если элемент листовой или его значение $\geq$ значение потомков, то конец;
	\item меняем местами значения элемента и его потомка, имеющего максимальное значение;
	\item выполняем процедуру просеивания вниз для изменившегося потомка. 
\end{enumerate}
\par\textbf{Вывод}
\par В данном разделе были рассмотрены алгоритмы сортировки пузырьком с флагом, выбором и пирамидальная сортировка.