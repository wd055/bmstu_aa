\chapter{Технологический раздел}
\label{cha:impl}
В данном разделе будут представлены листинги кода реализованных алгоритмов и дано описание получаемых результатов параметризации программы.

\section{Средства реализации}
В данной работе используется язык программирования Python. Среда разработки Visual Studio Code. Для замера процессорного времени используется функция process\_time() из библиотеки time.
\par Замеры времени были произведены на: Intel(R) Core(TM) i5-8250U CPU @1.60GHz 1.80 Ghz, 4 ядра, 8 логических процессоров.

\section{Листинг кода}
В листинге \ref{code:ets} приведен код алгоритма полного перебора.
\begin{lstlisting}[caption= Алгоритм полного перебора, label=code:ets]
class Graph():
	def __init__(self, _n, _c=None):
		self.nodes = _n
		self.c = _c
	
	def ets(self):
		start = [i for i in range(1, self.nodes)]
		min_len = float("inf")
		min_rout = [0 for _ in range(self.nodes)]
		routs = permutations(start)
		for rout in routs:
			rout = list(rout)
			rout.insert(0, 0)
			rout.append(0)
			cur_len = self.count_cost(rout)
			if cur_len < min_len:
				min_len = cur_len
				min_rout = rout
		return min_rout
\end{lstlisting}

\par В листинге \ref{code:anta} представлен код муравьиного алгоритма.
\begin{lstlisting}[caption= Муравьиный алгоритм, label=code:anta]
class Graph():
	def __init__(self, _n, _c=None):
		self.nodes = _n
		self.c = _c
	def ant_solve(self, herd, greed, evaportion, tmax):
		pheromones = [[0.1 for _ in range(self.nodes)] for _ in range(self.nodes)]
		visibility = [[1 / self.c[j][i] for j in range(self.nodes)] for i in range(self.nodes)]
		q = self.calculate_q()
		
		min_len = 0
		for i in range(1,self.nodes):
		min_len += self.c[i-1][i]
		min_len += self.c[self.nodes-1][0]
		min_rout = [i for i in range(self.nodes)]
		min_rout.append(0)
		
		possibilities = [i for i in range(self.nodes)]
		ants = [Ant(i,possibilities,herd,greed) for i in range(self.nodes)]
		
		for i in range(tmax):
			delta_pher = 0
			for i in range(self.nodes):
				ants[i].find_way(pheromones,visibility)
				l = self.count_cost(ants[i].rout)
				ph = q / l
				delta_pher += ph
				
				if l < min_len:
				min_len = l
				min_rout = [ants[i].rout[j] for j in range(len(ants[i].rout))]
				
				ants[i].reset(possibilities)
			
			for i in range(self.nodes):
				for j in range(self.nodes):
					if i != j:
						pheromones[i][j] *= (1 - evaportion)
						pheromones[i][j] += delta_pher
		
	return min_rout
\end{lstlisting}

\par В листинге \ref{code:ant} представлен код класса муравья.
\begin{lstlisting}[caption= Класс муравья, label=code:ant]
class Ant():
	def __init__(self, start, possibilities, _herd, _greed):
		self.possibilities = [possibilities[i] for i in range(len(possibilities))]
		self.possibilities.remove(start)
		self.base_pos = start
		self.pos = start
		self.herd = _herd
		self.greed = _greed
		self.rout = []
		self.rout.append(start)
	
	def find_way(self, pheromones, visibility):
		while 1:
			chances = []
			zn = 0
			for city in self.possibilities:
				t = pow(pheromones[self.pos][city], self.herd) * pow(visibility[self.pos][city], self.greed)
				zn += t
				chances.append([city, t])
			for i in range(len(self.possibilities)):
				chances[i][1] = chances[i][1] / zn
			
			city = self.make_choice(chances)
			self.pos = city
			self.possibilities.remove(city)
			self.rout.append(city)
			if len(self.possibilities) == 0:
				break
	
		self.rout.append(self.base_pos)
	
	def make_choice(self, chances):
		chances_sum = 0
		for chance in chances:
		chances_sum += chance[1]
		choice = uniform(0, chances_sum)     
		next_city = 0
		
		t = 0
		while choice > 0:
			choice -= chances[t][1]
			t += 1
		next_city = chances[t-1][0]
		return next_city
	
	def reset(self, possibilities):
		self.possibilities = [possibilities[i] for i in range(len(possibilities))]
		self.possibilities.remove(self.base_pos)
		self.pos = self.base_pos
		self.rout = [self.base_pos]

\end{lstlisting}

\par В листинге \ref{code:run} представлен код процесса параметризации.
\begin{lstlisting}[caption= Параметриазция, label=code:run]
def parametrization():
	g = Graph(10, matr1)
	g2 = Graph(10, matr3)
	
	reference1 = g.ets()
	reference_l1 = g.count_cost(reference1)
	
	reference2 = g2.ets()
	reference_l2 = g2.count_cost(reference2)
	
	f = open("parametrisation.txt", 'w')
	for herd in range(0, 10, 2):
		for greed in range(0, 10, 2):
			evaporation = 0.1
			while evaporation < 1:
				for days in range(10, 51, 10):
					f.write("{:2d} {:.1f} {:2d} ".format(herd, evaporation, days))
					
					rout1 = g.ant_solve(herd, greed, evaporation, days)
					rout2 = g.ant_solve(herd, greed, evaporation, days)
					l11 = g.count_cost(rout1)
					l12 = g.count_cost(rout2)
					tl1 = min(l11, l12)
					l1 = reference_l1 - tl1
					
					rout1 = g2.ant_solve(herd, greed, evaporation, days)
					rout2 = g2.ant_solve(herd, greed, evaporation, days)
					l21 = g2.count_cost(rout1)
					l22 = g2.count_cost(rout2)
					tl2 = min(l21, l22)
					l2 = reference_l1 - tl2
					
					f.write("{:2d} {:2d} {:2d} {:2d}\n".format(tl1, tl2, l1, l2))
				
				evaporation += 0.2
\end{lstlisting}

\section{Собираемые данные}
В результате параметризации формируется файл каждая строка которого -- комбинация параметров и полученные длины маршрутов и разность их и эталонных.

\section{Вывод}
В данном разделе были рассмотрены листинги кода реализованных алгоритмов и параметризации, описаны результаты работы программы.